% !TeX spellcheck = it_IT
\documentclass[10pt]{article}
\usepackage{amssymb,amsmath,amsfonts,xspace}
\usepackage[italian]{babel}
\usepackage[utf8]{inputenc}
\usepackage{graphics}
\pagestyle{empty}
\topmargin=-2.5cm
\oddsidemargin=-0.7cm
\textwidth=18cm
\textheight=25cm
\newcommand{\anac}{A.A. 2016/2017 }
\newcommand{\data}{6 novembre  2017}
\newcommand{\corso}{Complementi di Analisi Matematica}
\newcommand{\cdl}{Laurea Ingegneria Informatica e Automazione}
\newcommand{\esame}{Appello}
\newcounter{arc}
\newenvironment{exercise}
{\refstepcounter{arc}\begin{list}{\bf\thearc)}{\setlength{\leftmargin=0,9cm}\setlength{\labelsep=0,5cm}}\item}
        {\end{list}}
\newcommand{\exe}[2]{\begin{exercise} #1\hfill\flushright#2 pts. \end{exercise}}
\newcommand{\de}{\mathrm{d}}
\newcommand{\C}{\ensuremath{\mathbb{C}}\xspace}
\newcommand{\R}{\ensuremath{\mathbb{R}}\xspace}
\newcommand{\N}{\ensuremath{\mathbb{N}}\xspace}
\newcommand{\im}{\mathrm{i}}
\newcommand{\Log}{\ensuremath{\mathrm{Log}}\xspace}
\begin{document}
\begin{center}
\sf \large  Politecnico di Bari\\
\normalsize  \corso \\ \cdl\\
\anac \quad\quad \esame  \ \data \quad\quad Traccia A
\\ \vspace{1cm}
Cognome\hrulefill Nome\hrulefill N$^{\mathrm{o}}$ Matricola\hrulefill

\medskip
Programma:\mbox{}\hfill precedente AA 2014/2015 $\Box$
\hfill da AA 2014/2015 in poi $\Box$\hfill
 \vspace{0.5cm}
\end{center}

\vspace{1cm}

\exe{Enunciare e dimostrare una versione del teorema sulla trasformata di Laplace della derivata}
{5}


\setcounter{arc}{0}

\noindent {\em Per gli anni accademici precedenti al 2014/2015, si sostituisca l'esercizio 1) con il seguente:}





\exe{Dimostrare che se una serie di funzioni converge totalmente su un insieme $A$ allora  converge uniformemente su $A$}{5}


\exe{Determinare l'insieme di convergenza puntuale per la serie di potenze in \R:
	\[\sum_{n=1}^{+\infty}\frac{n\log n}{n^2+1} (x+1)^n.\]}{7}

\exe{Calcolare 
\[\int_{C^+(3,2)}\frac{\Log_0 z}{(z-3-i)^3}\de z,\] 
dove $C^+(3,2)$ \`e la circonferenza di centro $3$ e raggio $2$, orientata positivamente. }{6}

\exe{Enunciare e dimostrare il teorema di Hermite-Liouville.}{5}

\exe{Usando il metodo dei residui, calcolare 
	\[\int_{-\infty}^{+\infty}\frac{e^{-it}}{1+t^2}\de t.\]}{6} 

 
\exe{Calcolare la serie di soli seni della funzione 
$f(x)=x^2$, $x\in [0,1]$. Usando tale serie stabilire che 
\[\frac{1}{8}+4\sum_{h=0}^{+\infty} \frac{(-1)^h}{\pi^3(2h+1)^3}=\sum_{h=0}^{+\infty} \frac{(-1)^h}{\pi(2h+1)}. \]}
{7}
\end{document}