% !TeX spellcheck = it_IT
\documentclass[11pt]{article}
\usepackage{amssymb,amsmath,amsfonts,xspace}
\usepackage[italian]{babel}
\usepackage[utf8]{inputenc}
\usepackage{graphics}
\pagestyle{empty}
\topmargin=-1.5cm
\oddsidemargin=-0.5cm
\textwidth=17cm
\textheight=25cm
\newcommand{\anac}{A.A. 2015/2016 }
\newcommand{\data}{6 ottobre  2017}
\newcommand{\corso}{Analisi Matematica -- II modulo}
\newcommand{\cdl}{Laurea in Ingegneria Informatica e dell'Automazione}
\newcommand{\esame}{Appello}
\newcounter{arc}
\newenvironment{exercise}
{\refstepcounter{arc}\begin{list}{\bf\thearc)}{\setlength{\leftmargin=0,9cm}\setlength{\labelsep=0,5cm}}\item}
        {\end{list}}
\newcommand{\exe}[2]{\begin{exercise} #1\hfill\flushright#2 pts. \end{exercise}}
\newcommand{\de}{\mathrm{d}}
\newcommand{\C}{\ensuremath{\mathbb{C}}\xspace}
\newcommand{\R}{\ensuremath{\mathbb{R}}\xspace}
\newcommand{\N}{\ensuremath{\mathbb{N}}\xspace}
\newcommand{\im}{\mathrm{i}}
\newcommand{\Log}{\ensuremath{\mathrm{Log}}\xspace}
\begin{document}
\begin{center}
\sf \large  Politecnico di Bari\\
\normalsize  \corso -- \cdl\\
\anac \quad\quad \esame  \ \data \quad\quad Traccia A
\\ \vspace{1cm}
Cognome\hrulefill Nome\hrulefill N$^{\mathrm{o}}$ Matricola\hrulefill
%\medskip
% Programma:\mbox{}\hfill AA 2009/2010 $\Box$
% \hfill AA 2012/2013 $\Box$\hfill Altri anni $\Box$\hfill
% % \vspace{1cm}
\end{center}

\medskip
\exe{\begin{itemize}
		\item Calcolare la somma della serie 
		\[\sum_{n=3}^{+\infty}\frac{(e-2)^n}{e}.\]
		\item Stabilire il carattere della serie 
		\[\sum_{n=2}^{+\infty}\frac{\cos n-2}{n^{3/2}-n}.\]
	\end{itemize}
}{6}              

\exe{Determinare il dominio della funzione reale di due variabili reali 
	\[f(x,y)=(x^2y-e^{\sqrt{x^2-y^2}})^{1/3}\] e rappresentarlo graficamente sul piano. 
	Stabilire poi che $f$ ha derivata direzionale nel punto $(1,0)$ secondo il versore $v=\left(-\frac{1}{\sqrt 2},-\frac{1}{\sqrt 2}\right)$. Calcolare  quindi 
	$\frac{\partial f}{\partial v}(1,0)$.}{8}
\exe{Determinare la soluzione del problema di Cauchy
	\[	\begin{cases}
	y''+y'=2e^{-x}-x\\
	y(0)=1\\
	y'(0)=0
	\end{cases}\]
}{8}


\exe{Calcolare \[\int_A \frac{2xy}{x^2-y^2}\de x\de y,\]
dove $A=\{(x,y)\in\R^2:0\leq x\leq 1, \ 0\leq y\leq \dfrac{x}{2}\}$.}{8}


%
%\newpage
%\setcounter{arc}{0}
%
%
%
%
%\begin{center}
%\sf \large  Politecnico di Bari\\
%\normalsize  \corso -- \cdl\\
%\anac \quad\quad \esame  \ \data \quad\quad Traccia B
%\\ \vspace{1cm}
%Cognome\hrulefill Nome\hrulefill N$^{\mathrm{o}}$ Matricola\hrulefill
%\end{center}
%
%\medskip
%\exe{Stabilire se la funzione $f(x)=\dfrac{\sin(x^4)}{x\log^{4/3}x} $ \`e integrabile tra $0$ e $\frac{1}{3}$.
%}{7}              
%
%
%\exe{Determinare la soluzione del problema di Cauchy:
%	\[\begin{cases}
%	y'=\frac{x^2}{x^2+1}y\\
%	y(0)=1
%	\end{cases}\]
%Si consideri poi la seguente modifica del problema precedente 
%\[\begin{cases}
%	y'=\frac{x^2}{x^2+1}y+\cos y\\
%	y(0)=1
%	\end{cases}\]
%\`E ancora vero che tale problema ha una ed una sola soluzione definita su \R? Motivare la risposta.}{8}
%
%\exe{Determinare i punti stazionari della funzione 
%	\[f(x,y)=(1-x^2+y^2)xy\]
%	e studiarne la loro natura.}{8} 
%
%
%\exe{Calcolare 
%	\[\int_A \frac{x^2\cos x}{\cos^2(xy)}\de x\de y,\]
%	dove $A=\{(x,y)\in\R^2:\frac{\pi}{6}\leq x\leq \frac{\pi}{4},\ 1\leq y\leq\frac{\pi}{4x}\}$.
%	}{7}
%
%
\end{document}